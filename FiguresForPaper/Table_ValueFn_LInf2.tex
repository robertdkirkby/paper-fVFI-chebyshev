\begin{tabular*}{1.00\textwidth}{@{\extracolsep{\fill}}l|ccccccc} 
 \hline \hline 
 Order \textbackslash Order & 1 & 2 & 3 & 4 & 5 & 7 & 9 \\ 
 \hline 
1 & 2.43e-01 & 1.68e-01 & 1.68e-01 & 1.68e-01 & 1.68e-01 & 1.68e-01 & 1.68e-01 \\ 
 2 & 1.60e-02 & 1.60e-02 & 1.60e-02 & 1.60e-02 & 9.76e-03 & 9.76e-03 & 9.76e-03 \\ 
 3 & 1.60e-02 & 1.60e-02 & 1.60e-02 & 1.60e-02 & 1.41e-03 & 1.41e-03 & 1.41e-03 \\ 
 4 & 1.60e-02 & 1.60e-02 & 1.60e-02 & 1.60e-02 & 1.41e-03 & 1.41e-03 & 1.41e-03 \\ 
 5 & 9.02e-03 & 3.79e-03 & 1.41e-03 & 1.41e-03 & 1.41e-03 & 1.41e-03 & 1.41e-03 \\ 
 7 & 9.02e-03 & 3.79e-03 & 1.41e-03 & 1.41e-03 & 1.41e-03 & 1.41e-03 & 1.41e-03 \\ 
 9 & 9.02e-03 & 3.79e-03 & 1.41e-03 & 1.41e-03 & 1.41e-03 & 1.41e-03 & 1.41e-03 \\ 
 \hline 
 \end{tabular*} 
\begin{minipage}[t]{1.00\textwidth}{\baselineskip=.5\baselineskip \vspace{.3cm} \footnotesize{ 
Notes: $L_{\infty}$ norm measure of difference between Value function (for Stochastic Neoclassical Growth Model) and the Smolyak-Chebyshev approximation of given orders (the Smolyak level of approximation is set to the rounded up value of the square-root of the largest Chebyshev order; e.g., if the Chebyshev polynomial orders are 5 and 7, then Smolyak level of approximation is 3, as sqrt(7)=2.65). Rows are the order of the Chebyshev polynomial on physical capital dimension, columns the productivity shock dimension. \\ 
}} \end{minipage}